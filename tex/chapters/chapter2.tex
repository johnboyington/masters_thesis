\cleardoublepage

\chapter{Review of Available Methods and Data}


%%%%%%%%%%%%%%%%%%%%%%%%%%%%%%%%%%%%%%%%%%%%%%%%%%%%%%%%%%%%%%%%%%%%%%%%%%%%%%%%
\section{Beam Port Characterization Experiments}

%% introduction
%When measuring charged particles, the energy is directly measurable, which is nice.
A useful effect when measuring charged particles and gamma-rays is that generally, it is possible for all of the particle's energy deposited into a detector within a relatively small time window.
%You can't do that with neutrons because they are neutral particles, which is less nice.
Neutrons, unfortunately, do not share this property with the other particle types, making the process of obtaining spectral information about some neutronic environment less straightforward.
% they generally have to be measured indirectly - a neutron reacts with something causing something else that is measureable
Instead, neutrons have to be measured indirectly - a neutron will react and produce some measureable secondary effect within a detection medium.
This effect is rarely equal to (or even proportional to) the energy of the incident neutron.
That does not mean, however, that spectral information cannot still be obtained about a neutron flux.
% several of these indirect interactions can be compiled to give information on a whole neutron spectrum
In practice, with a set of different detectors, each responding to different speeds of neutrons, a collection of measurements can be compiled to produce information on the entire spectrum.
% here we will explore two common ways neutron spectra are investigated
Here, we will explore two common ways neutorn spectra are investigated experimentally in practice.

\subsection{Bonner Sphere Spectrometer}
%% bonner spheres
% a major breakthrough came in 1961 when Brablett et. al. introduced the idea of bonner spheres
A major breakthrough came for the neutron dosimetry community in the early 1960s when Bramlett et. al. first introduced their ``new type of neutron spectrometer.''
% they developed an active neturon spectrometer
The ``Bonner Sphere'', named after one of the original Rice Univeristy authors, was a first-of-its-kind, active neutron spectrometer.
% here's how it works
The device allows measurements to be taken across the entire range of energies found within or around a nuclear reactor or other common neutron source.
% there's a thermal sensitive crystal at the bottom
Fundamental to the design is the thermal-sensitive LiI crystal that serves as the detection medium.
% first, a measurement is taken in the neutron environment
First, a measurement is taken using the bare sensor.
% then, the smallest sphere is added
Then, a small High Density Polyethylene sphere is placed around the crystal.
% adding this sphere does two things, absorbs thermal neutrons and downscatters faster neutrons so they can be more readily absorbed
The purpose of this is twofold - the sphere will remove a portion of thermal neutrons from the system and also will downscatter faster neutrons so they can be more readily absorbed by the crystal.
% following the mesaurement with the sphere on, larger and larger spheres are used to produce faster and faster responses
Following this measurement, larger and larger spheres will be used to produce faster and faster responses.
% after taking all of these measurements, the entire spectrum has been measured
After taking all of these measurements, information will have been obtained from the entire neutron spectrum.
% using some math (which will be detailed in the next section) these measurements can be used to obtain the actual spectrum
Using some mathematical techniques (which will be detailed in the next section) these measurement can be used to finally obtain a full, high resolution neutron spectrum.

\subsection{Gold Foil Activation}
%% gold foil activation
% foil activation, specifically with gold foils, is a standard way of measuring flux
Foil activation, specifically through the use of gold foils, is a standard way of measuring a neutron flux.
% how does it work
The foils (often wires are used, too) are placed into the neutron flux.
Neutrons activate the foils causing them to become radioactive and emit gamma-rays.
These gamma-rays can be measured, and the equations that govern the activation process can be used in reverse to back out the incident flux.
% with the gold alone, only one response can measure the absolute flux
With a gold foil alone, a single response can be obtained which yields the absolute flux of a system.
% gold is good for this for a number of reasons
Gold, in particular, is useful for this for a number of reasons.
% very high thermal cross section, it has a great halflife and gamma for counting and is metallic, so it can be worked into different forms
First, its $(n, \gamma)$ reaction has a very high thermal cross section.
It also has a 2.7 day halflife, which is ideal, for it will not quickly decay away, yet it also doesn't take an unreasonable amount of time to reach a substantial fraction of saturation.
The gamma-ray produced by gold is approximately 412 keV, ideal for measuring with an HPGe detector.
It is also metallic, meaning that it can be easily worked into different forms depending on the desired measurement.
% though, gold alone cant give spectrum, there are a number of ways in which you can get one
Though a single gold foil cannot give information on the spectrum, there are a number of ways in which gold can be used to help measure the entire spectrum.
% adding cadmium allows one to get a two group flux, since cadmium eats thermal neutrons
Adding a cadmium filter around the gold will produce a two-group flux, since cadmium will absorb all of the thermal neutrons and isolate gold's major resonance.
% gold can be used as a part of a multifoil experiment
Gold can also be used as part of a multi-foil experiment.
Different kinds of foils with different responses can be used to cover the whole spectrum.
% it has also been used with the bonner spheres to produce info from the whole spectrum
Also, gold has been used with a set of bonner spheres, acting as the primary detection medium within the center of the spheres.
% in beam, this works
This has been done in-beam at many different research and medical isotope production reactors.
