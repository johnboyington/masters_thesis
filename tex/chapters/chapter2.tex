\cleardoublepage

\chapter{Review of Available Methods and Data}


%%%%%%%%%%%%%%%%%%%%%%%%%%%%%%%%%%%%%%%%%%%%%%%%%%%%%%%%%%%%%%%%%%%%%%%%%%%%%%%%
\section{Beam Port Characterization Experiments}

(Placeholder text)
When measuring charged particles, the energy is directly measurable, which is nice.
You can't do that with neutrons because they are neutral particles, which is less nice.
Instead, most modern methods use discrete responses of individual detectors or detector configurations.

Will talk about Bonner Spheres and Foil Activation here.

%%%%%%%%%%%%%%%%%%%%%%%%%%%%%%%%%%%%%%%%%%%%%%%%%%%%%%%%%%%%%%%%%%%%%%%%%%%%%%%%
\section{Spectral Unfolding Methods}


% ------------------------------------------------------------------------------
\subsection{Formulation}


In practice, active and passive detectors and measurement techniques can appear very different.
However, for the purpose of mathematical formulation, it is possible, and indeed useful, to abstract their similitudes.
The above mentioned detection methods are resemblant in the fact that they provide a set of unique, discrete responses in the presence of an unknown, continuous neutron flux.
These discrete responses can vary greatly between different neutron environments.
For example, the LiI detection crystal of the BSS responds strongly in the presence of thermal neutrons and shows indifference towards neutrons of the faster variety, whereas certian reactions utilized in activation foil analysis, such as ($n, \alpha$), will not occur below specific, fast, threshold neutron energies, thus the detector is considered to have a fast response.
These largely energy-dependent responses, often related to material properties and reaction cross sections, can be considered functions.
A detector's `Response Function', now formally stated, is the measureable effect exhibited by a detector in a particular geometric configuration as a function of energy to a particular neutron source.
It is true that the response function is more technically also a function of parameters like geometry, source angle and position, etc.; however, it proves efficacious here to hold those details constant and consider response unidimensional in energy.

This conceptualization of response allows for a stated relationship between flux, response function and response for detector configuration $k$, shown in the following Fredholm integral equation of the first kind,

\begin{equation}\label{eqn:cont-response}
N_k + \epsilon_k = \int_E R_k(E) \phi(E) dE .
\end{equation}

Here, $N_k$ represents the measured response of the detector, $\epsilon_k$ is the unknown error associated with $N_k$, $R_k(E)$ is the response function, and $\phi(E)$ is the flux.
By placing a detector, $k$, with associated response function, $R(E)$, in a neutron environment, $\phi(E)$, it should exhibit some response $N_k$ that is dispaired $\epsilon_k$ from the expected response.

As posed, this formulation bears a certain number of issues.
It is impossible to derive a continuous function from a finite set of equations.
With any discrete number of measurements, the solution remains non-unique; the equation thus satisified by an infinite number of solutions.
The fact that the $\epsilon_k$ term is unknown implies that just as there is a limit to the confidence one can have in a real measurement, there too, is a limit on the intelligibility of the final spectrum.

A number of approximations serve to remedy these dilemmata.
An energy bin structure can be introduced, with energy bins $\Delta E_i (i = 1, \ldots, n)$, which discretize the problem.
Then, for a problem with $m$ detector configurations, equation \ref{eqn:cont-response} is rewritten as

\begin{equation}\label{eqn:disc-response}
N_k + \epsilon_k = \sum_i R_{ki} \phi_i, \quad k = 1,\ldots, m .
\end{equation}

Response functions are generally computed, using either deterministic or Monte Carlo methods, and are therefore more naturaly accomodated by this equation.
The unknown $\epsilon_k$ term can be handled by removing it entirely, although more sophisticated approaches to solving will utilize a known variance, $\sigma^2$, related to the measurement environment, into the analysis and propogate this error into the final $\phi$.


% ------------------------------------------------------------------------------
\subsection{Proposed Methods of Solution}

% here, i mention the ammount of solution methods and then explain why there is a problem with a math-only approach
% there's a lot of methods
The process of obtaining the solution for the unknown neutron spectrum from the set of possible solutions is known as spectral deconvolution (or unfolding), and given the number of valid solutions to the ill-posed problem, the volume of existing deconvolution methods is, perhaps unsurprisingly, commensurate to the number of solutions themselves.
% it is possible to just use a math-based approach
% lots of math based methods exist that give acceptable solutions
% however, these can produce spectra with features that are unreasonable
Solution spectra based on pure-math approaches could lack or include features that aren't reasonable, because the solution is meant to represent a *REAl* phenomenon.
% this claim can be made because much is already known about the physics in a reactor core
It's therefore necessary to incorporate some of these physics into the method of solution.
% reactors have been studied for a long time
% this has led to the development of a strong theory of reactor theory and neutronics
% this knowledge is crutial for discriminating bad or un-physical spectra, without it, there'd be know way to reasonablly accept a solution
% this means that math only approaches are no good
% to have a good solution method, we need to incorporate this information into our analysis

% it is common practice to analize the spectrum after unfolding to see if it appears physically reasonable
% the features in the spectrum should match that of our understanding
% however, this leaves the problem of infinity solutions, and doesn't assist the liklihood that we'll arrive on a reasonable solution

% constraints can be added and the problem viewed as an optimizaiton problem
% examples of these constraints include avoiding jumps, and other stuff, and these will be discussed later

% by far, the most common approach to this dilema is the inclusion of what is called a default spectrum
% this spectrum, also known as a guess or trial spectrum, is the best initial guess for what the final spectrum could be
% it makes sense, since ya know, perfect guess, then we've already got the answer
% if the trial spectrum is slightly off, our solution method just has to tweak it slightly to arrive at a solution
% this changes the view of unfolding from a ground up production of a solution, to a tweaking process

% but can we do this?
% it's the a priori understanding we've got of reactor cores that allow us to do this
% taking for example a typical lwr, a spectrum should generally consist of a superposition of 3 features
% these features are the fission neutron spectrum, the 1/e region and the maxwellian distribution
% these all can conveniently be represented mathematically, however, the exact values of certain constants are unknown
% using this as a default spectrum input to an unfolding method, the spectrum can be tweaked to match the detector measurement and the solution is almost guaranteed to be physically reasonable, as it won't be a huge departure from something that is *by defintion* reasonable

% do we want more accuracy?
% we also might not be in a lwr reactor core?
% even if we're not in one of these locations, we don't have to worry, numerical simulations often can provide us with a guess for the neutron spectrum
% these go beyond the scope of just regular analytical solutions
% modern computers with lots of computing power allow for the modeling of incredibly complex physical features
% both MC and deterministic methods can be applied to a neturonic problem and obtain a neutron spectrum


The most commonly applied version of this is by using an estimate for the final spectrum.
Codes refer to this as either the guess, trial, or default spectrum.
This spectrum 



%                   -------------------------------------

\subsubsection{MAXED}




%                   -------------------------------------

\subsubsection{Gravel}

%                   -------------------------------------

\subsubsection{STAY'SL}




%%%%%%%%%%%%%%%%%%%%%%%%%%%%%%%%%%%%%%%%%%%%%%%%%%%%%%%%%%%%%%%%%%%%%%%%%%%%%%%%
\section{Variance Reduction Methods}

