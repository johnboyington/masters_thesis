\cleardoublepage

\chapter{Review of Available Methods and Data}


%%%%%%%%%%%%%%%%%%%%%%%%%%%%%%%%%%%%%%%%%%%%%%%%%%%%%%%%%%%%%%%%%%%%%%%%%%%%%%%%
\section{Beam Port Characterization Experiments}

(Placeholder text)
When measuring charged particles, the energy is directly measurable, which is nice.
You can't do that with neutrons because they are neutral particles, which is less nice.
Instead, most modern methods use discrete responses of individual detectors or detector configurations.

Will talk about Bonner Spheres and Foil Activation here.

%%%%%%%%%%%%%%%%%%%%%%%%%%%%%%%%%%%%%%%%%%%%%%%%%%%%%%%%%%%%%%%%%%%%%%%%%%%%%%%%
\section{Spectral Unfolding Methods}


% ------------------------------------------------------------------------------
\subsection{Formulation}


In practice, active and passive detectors and measurement techniques can appear very different.
However, for the purpose of mathematical formulation, it is possible, and indeed useful, to abstract their similitudes.
The above mentioned detection methods are resemblant in the fact that they provide a set of unique, discrete responses in the presence of an unknown, continuous neutron flux.
These discrete responses can vary greatly between different neutron environments.
For example, the LiI detection crystal of the BSS responds strongly in the presence of thermal neutrons and shows indifference towards neutrons of the faster variety, whereas certian reactions utilized in activation foil analysis, such as ($n, \alpha$), will not occur below specific, fast, threshold neutron energies and the detector is considered to have a fast response.
These largely energy-dependent responses, often related to material properties and reaction cross sections, can be considered functions.
A detector's `Response Function', now formally stated, is the measureable effect exhibited by a detector in a particular geometric configuration as a function of energy to a particular neutron source.
It is true that the response function is more technically also a function of parameters like geometry, source angle and position, etc.; however, it proves efficacious to hold those details constant and consider response unidimensional in energy.




% ------------------------------------------------------------------------------
\subsection{Proposed Methods}


%                   -------------------------------------

\subsubsection{MAXED}




%                   -------------------------------------

\subsubsection{Gravel}

%                   -------------------------------------

\subsubsection{STAY'SL}




%%%%%%%%%%%%%%%%%%%%%%%%%%%%%%%%%%%%%%%%%%%%%%%%%%%%%%%%%%%%%%%%%%%%%%%%%%%%%%%%
\section{Variance Reduction Methods}

