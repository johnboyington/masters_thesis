
\cleardoublepage


\chapter{Northeast Beam Port Modeling}


\section{Application of Advantg}

% okay, we've got an SDEF problem, what kind of tallies would we like?
After obtaining the fission rate tallies and creating an appropriate SDEF card, an MCNP problem formulation is now only missing a beam port tally.
The tallies are setup here to provide the maximum detail and understanding of what is happening at the outer aperture of the NEBP.
First, the outer face of the NEBP is used in a surface tally (F2) to find the fluence departing.
This tally was divided spatially using different concentric cylinders, providing information on seven regions of the NEBP.
The first three are all equivalent area tallies within the collimator itself.
This will provide an understanding of the radial distribution with the main beam.
The fourth segment was the face of the aluminum collimator.
Then, the last three were equivalent area tallies within the BP collimator.
This would give an idea for how neutrons managed to escape the beam but still contribute to the fluence that leaves the beam.

% we've also got them divided into cosine and energy
Then, both angular and spectral groupings were used to bin each of these regions.
For the angle, a custom, fine, 109-group structure was used.
The first 20 bins were tenth degree increments between 0$^{\circ}$ and 2$^{\circ}$.
Following that, a resolution of one degree were used between 2$^{\circ}$ and 90$^{\circ}$
All of the backward fluence was captured by one bin between 90$^{\circ}$ and 180$^{\circ}$.

For the energy structure, the scale252 group structure was used.
This medium-resolution structure was selected because of the trade off between wanting to capture finer spectral details and the run-time to resolve statistics.

% great, can we run this?
With the tally now in place, MCNP was run, first without any variance reduction (VR).
Unsurprisingly, only a handful of the several million simulated particles survived to the end of the NEBP and were captured by the tally.
It became quickly apparent that this problem was in need of some serious form of VR to be able to achieve decent tally results in any reasonable amount of time.

% let's pick advantg.
For this application, the code ADVANTG was select to provide the VR needed to do the transport.

% but what is advantg?
% i will talk about it a bit here
% then i'll explain why it's useful for our problem

% how did i apply it to our problem

% first, we rotated the model
% that's because since advantg uses a mesh, we can set the bp on axis, which is much better

% then here, i'll just talk about all of the advantg parameters and why they were used
% basically, everything comes down to runtime, memory, and FOM

% so i ran it, that gave me a wwinp file that was how many MB?

% let's put pics here of the wws and talk about them
% like, quite a bit of discussion here is warranted

% then, talk about some of the modifications to the source term and stuff.
% i think that's probably good for advantg
