
\cleardoublepage


\chapter{Conclusions and Future Work}

\section{Conclusions}
%% concise statements about your main findings, related to your aims/objectives/hypothesis.
% it's possible to simulate the beam w/ advantg
First, through the assistance of the VR methods described, it is possible to find the neutron spectrum for a collimated reactor beam port.
% a high resolution flux was obtained via simulation
A high resolution flux spectrum was obtained via Monte Carlo simulation for the KSU TRIGA Mark II's Northeast Beam Port featuring dimensional data in energy, angle, and the radial direction.
% two separate experiments showed data that agreed with the shape of the simulation work and showed similar magnitude differences
Two separate experiments were conducted which collected data that demonstrated decent shape-agreeance with the simulated spectrum and differed in magnitude from the simulated work by similar values.
% the foil tube was novel
Of the two experiments, one involved the creation of a novel type of neutron spectrometer which solved issues like beam alignment and allowed for several different responses to be obtained using only one irradiation and one type of foil (gold).
% successful tuning of spectrum led to a set of solution spectra that is consistent with experimental data
Successful deconvolution using the experimental data led to a set of solution spectra that is consistent with a typical reactor spectrum shape, the original simulated spectrum shape, and the experimental data.
% the shape looks softer than originally hypothesized, resembling an in-core neutron flux
The final shape of the spectrum doesn't match the original hypothesis of being predominately fast, but rather resembles an in-core neutron flux.
% the beam is functionally monodirectional
The final flux is approximately monodirectional.
% a significant majority of the flux is departing from inside the collimator
Also, a significant majority of the flux departing from the beam port lives with the void of the collimator.



\section{Contributions}
%% stating/restating the significance of what you have discovered. Can include limitations
% through this analysis we know a lot more about the beam port
Through this analysis, a much more complete picture of the NEBP has been obtained.
% this analysis replaces the old data allowing experimentalists a better understanding of the environment
This replaces the sparse existing data, which allows experimentalists to better understand the environment that their devices will be tested in.
% the set of experiments provides a benchmark for future simulation
The set of experiments conducted provides a set of benchmarks for future simulations of the NEBP.
% this effort led to additions to the ksu reactor model making it more complete
This effort led to several additions to the KSU reactor model, making it more complete.
% steps in this analysis could easily be extended to any reactor beam problem
Many of the steps taken in this analysis could be easily extended to other reactor beam problems, and the success found here allows other analysts the confidence to invest time attempting to simulate their beam in this way.
% the foil tube design is a easy to manufacture device that can be used as a new way to characterize any neutron beam
The foil tube design used here is an easy to manufacture device that can be adapted and used as a new way to characterize a neutron beam.



\section{Future Research}
%% where to go from here (can include where NOT to go, if your research demonstrated that a particular approach or avenue was not useful).
% the other three beam ports should be simulated
Following this work, the other three beam ports at KSU can be characterized in a similar way.
% the framework provided here will extend easily to the other beam ports
The framework provided here will easily extend to those other beam ports and allow the analyses to be done much quicker.
% this includes the experimental methods
This includes the experimental methods utilized.
% work could also be done to study the difference in power/temperature effects as well as the effect control rods may have on the beam port spectrum
Work can also be done to study the difference made on the spectrum by operating the reactor a different powers/temperatures, as well as the effect that different control rod positionings could have on the spectrum.
% the experiments done here should be redone regularly, to consistently benchmark the spectrum
The experiments conducted in this research should be redone regularly to provide a continually evolving understanding of the NEBP spectrum.

% the foil tube alone provides a framework for more research
The foil tube desing alone also provides a framework for more research to be conducted.
% the design could be done with different kinds of foils, and become an optimization problem for spacers, foil types, the inclusion of cadmium
There are many different parameters the device could be optimized for.
This includes the spacer widths and materials, different or multiple foil types, the inclusion of cadmium or some other covering material in the design, and different overall geometric concepts.



