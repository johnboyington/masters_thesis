
\cleardoublepage


\chapter{Experimental Validation}



\section{Postprocessing and Results}


\subsection{Gold Foil Tube}


% postprocessing includes taking the measured activities and converting them to saturation activities
The postprocessing necessary for a foil activation experiment involves taking the measured activities, which are reported from the gamma spectrometry software, and backing out the saturation activities for each foil.
% this is done using the following equation
This is done using the following equation,

% how to get saturation activities
\begin{equation}
\label{eqn:a_sat}
A_{sat} = A_{meas} \frac{R_{meas} R_{sat}}{n_a K I_{rel}} ,
\end{equation}

% where, (explain these terms)
where $R_{meas}$ is the ratio that corrects for decay during measurement, $R_{sat}$ is the saturation ratio which will be detailed below, $n_a$ is the number of sample atoms, equal to $\rho N_A / M$, $K$ is the isotopic abundance, and $I_{rel}$ is the relative intensity of the counted gamma ray.

% give the term and explain for r_meas
The term, $R_{meas}$ is the ratio of the activity at the beginning of the gamma counting to the activity assuming the decay is constant during counting.
It is given by the following equation

% measurement ratio/correction factor
\begin{equation}
\label{eqn:r_meas}
R_{meas} = \frac{\lambda t_{meas}}{1 - e^{\lambda t_{meas}}},
\end{equation}

in which $\lambda$ is the decay constant and $t_{meas}$ is the time during measurement.


% explain the concept of saturation ratio
% this is the big one

\subsection{Bonner Sphere Spectrometer}
