
\cleardoublepage


\chapter{Experimental Validation}



\section{Postprocessing and Results}


\subsection{Gold Foil Tube}


% postprocessing includes taking the measured activities and converting them to saturation activities
The postprocessing necessary for a foil activation experiment involves taking the measured activities, which are reported from the gamma spectrometry software, and backing out the saturation activities for each foil.
% this is done using the following equation
This is done using the following equation,

% how to get saturation activities
\begin{equation}
\label{eqn:a_sat}
A_{sat} = A_{meas} \frac{R_{meas}}{R_{sat} n_a K I_{rel}} ,
\end{equation}

% where, (explain these terms)
where $R_{meas}$ is the ratio that corrects for decay during measurement, $R_{sat}$ is the saturation ratio which will be explained below, $n_a$ is the number of sample atoms, equal to $\rho N_A / M$, $K$ is the isotopic abundance, and $I_{rel}$ is the relative intensity of the counted gamma ray.

% give the term and explain for r_meas
The term, $R_{meas}$ is the ratio of the activity at the beginning of the gamma counting to the activity assuming the decay is constant during counting.
It is given by the following equation

% measurement ratio/correction factor
\begin{equation}
\label{eqn:r_meas}
R_{meas} = \frac{\lambda t_{meas}}{1 - e^{\lambda t_{meas}}},
\end{equation}

in which $\lambda$ is the decay constant and $t_{meas}$ is the time during measurement.

% explain the concept of saturation ratio
The saturation ratio, $R_{sat}$, is the ratio between the activity at the time of measurement and the saturation activity.
Although this value generally assumes constant production during irradiation followed by a period of decay with no production, a method has been developed that captures transient behavior in reactor fluxes which will be detailed here.

The production and decay of radioisotopes is expressed mathematically through the bateman equation,

\begin{equation}
\label{eqn:bateman}
\frac{dN(t)}{dt} = C(t) - \lambda N(t)
\end{equation}

where $N(t)$ is the number of radioactive nuclei in the sample, C(t) is the production term in nuclei per second, and $\lambda$ is the decay constant for the isotope in consideration.
The change in radionuclei is simply the production minus the loss.
The relationship between activity and radionuclides, $A(t) = \lambda N(t)$ allows us to convert \EQ{eqn:bateman} into the following,

\begin{equation}
\label{eqn:bateman_activity}
\frac{A(t)}{dt} = \lambda (C(t) - A(t)).
\end{equation}

We can then divide each of the terms by $C_{sat}$ which is the isotopic production at nominal power.
This value is equal to the saturation activity, since the activity at $t = \inf$ for nominal power is equal to the production at that power.

\begin{equation}
\label{eqn:bateman_ratios}
\frac{A(t) / C_{sat}}{dt} = \lambda (\frac{C(t)}{C_{sat}} - \frac{A(t)}{C_{sat}})
\end{equation}

Two facts allow us to simplify \EQ{eqn:bateman_ratios} into a useful form.
First, because $A_sat = C_sat$, any $A(t) / C_{sat} = A(t) / A_{sat}$, so we can replace these terms with a defined term, $R_{sat}(t)$ which is the ratio of the activity at time $t$ to the saturation activity.
Second, although we don't have access to the $C(t)$ or $C_{sat}$ terms since they require the (currently unknown) flux, we do have access to power data.
Because isotopic production is approximately proportional to reactor power, $C(t) \propto P(t)$, this means that $C(t) / C_{sat} \approx P(t) / P_{nominal}$.
We will replace this power ratio with the term $P_{f}(t)$, which is simply the time dependent ratio of the power to the nominal power.

\begin{equation}
\label{eqn:bateman_r_sat}
\frac{R_{sat}(t)}{dt} = \lambda (P_{f}(t) - R_{sat}(t))
\end{equation}

This equation can be solved to find $R_{sat}$ at the time of measurement using a numerical solver, in our case, {\tt scipy}'s {\tt odeint}.
The data for $P_{f}(t)$ is simply normalized strip chart data collected from an in-core fission chamber.


% ---------------------
% fix this table later
% ---------------------

% this contains all of the correction factors and foil activities
\begin{table}[h]\centering
\label{tab:a_sat}
\caption{The correction factors and foil activities.}
\begin{tabular}{ r | r | r | r | l }
\toprule
Position (in.)  & $A_{meas}$  & $R_{meas}$  & $R_{sat}$   &   $A_{sat}$\\
0 & 1.48373626e+03 & 1.00089748 & 0.02268378 & 65468.28629567 \\
1 & 5.86027460e+01 & 1.00590363 & 0.02262131 &  2605.89285917  \\
2 & 8.92905830e+00 & 1.03805564 & 0.02196285 &   422.02438091  \\
3 & 4.06821290e+00 & 1.06581552 & 0.0204511  &   212.01617444  \\
4 & 2.58165872e+00 & 1.14023228 & 0.01574428  &   186.96886973  \\
5 & 1.60404768e+00 & 1.2797683 & 0.01340611 &      153.12486185  \\
6 & 7.21675010e-01 & 1.26894006 & 0.00794792 &       115.22035675  \\
7 & 3.68363379e-01 & 1.27340132 & 0.00483425 &      97.0315587   \\
8 & 1.73328239e-01 & 1.4158706 & 0.00256568 &       95.65112019\\
\end{tabular}
\end{table}


\subsection{Bonner Sphere Spectrometer}





