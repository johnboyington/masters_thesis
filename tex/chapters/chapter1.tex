
\cleardoublepage


\chapter{Background and Introduction}



\section{Neutron Spectrometry}
%% state the general topic and give some background
% capabilities of nuclear technology is very powerful
In the last century, nuclear technology has proven itself to be a field of tremendous power and utility.
% however, these great capabilities are often limited by our true understanding of the systems that underly them
However, these great capabilities are often limited by our true understanding of the neutronic systems that underlie them.
% having methods to obtain accurate knowledge of how these systems operate and evolve raise this fundamental bar
Having proper methods to accurately obtain knowledge of how these systems operate and evolve raises this fundamental limit on our capabilities.
% therefore, from essentially the beginning of nuclear science, measurement techniques have had an __ niche in supporting the development of the science and technology
% and today this is no different
Therefore, from the beginning of nuclear science, measurement techniques have had an indispensable niche in supporting the development of these technologies, and today, that is no different.


%% provide a review of the literature related to the topic
% in the field of neutron detection, a huge body of scientific methods and technologies have developed that allow an experimentor to reliably and accurately obtain a most fundamental property of ___, the neutron spectrum
In the field of neutron detection, a huge body of scientific methods and technologies have developed that allow an experimenter or analyst to reliably and accurately obtain a most fundamental property of these systems, the neutron flux spectrum.
% active scintillation detectors, proton recoil detectors, and passive activation detectors represent a portion of the different way information on a neutron spectrum is obtained
Active scintillation spectrometers and passive foil or wire activation detectors represent a portion of the different ways information on a neutron spectrum can be obtained.
% though these detectors allow us to find information on a neutronic system, a large field of mathematics have allowed even greater capabilities and even more information to be extracted
Though these detectors allow us to find information on a neutronic system, a large field of mathematics have allowed even greater understanding by allowing even more information to be extracted.
% spectral deconvolution/unfolding techniques have allowed a small series of measurements to be translated into a very large group neutron flux, a feat once considered impossible
Spectral deconvolution techniques have allowed a small series of measurements to be translated into very large group neutron fluxes, a feat once considered impossible.

%% define the terms and scope of the topic
% this field, neutron spectrometry, represents the field of research, within which the topics discussed in this thesis lie.
This field, neutron spectrometry, represents the field of research within which the topics discussed in this thesis lie.
% by employing a variety of simulation, measurement, and unfolding techniques, the author will show how these components have fit together to ultimately widen the field of scientific knowledge
The author will show how simulation, measurement, and deconvolution techniques have been fit together in a specific application to ultimately widen the current field of scientific knowledge.



\section{Current State of the KSU TRIGA Mark II's Northeast Beam Port}
%% outline the current situation
% mention the mark ii reactor
In our case, the neutronic system in question is Kansas State University's TRIGA Mark II research reactor.
% mention the beam ports
Four beam ports penetrate the reactor structure and allow neutrons to stream from the reactor core to the outside for the conduction of ex-core irradiations.
% initial flux characterization f/t ratio, flux
Upon building the reactor, an initial flux characterization of these ports was conducted, reporting the magnitude of the flux, as well as the fast-to-total ratio of these ports \cite{ksutrainingmanual}.
% since then, geometry of the beam port has changed significantly
Separate works have been published that further characterized the flux in one of the beam ports, the northeast beam port, \cite{bouchey1967experimental} \cite{ryan1998analysis}; however, since then, a collimator has replaced the wooden plugs that previously occupied the northeast beam port, significantly changing its geometry.
% the core has also underwent change, fuel rod positions, burnup
The reactor core itself has also underwent change, including burnup and the shuffling/replacement of fuel elements.

%% evaluate the current situation (advantages/disadvantages) and identify the gap
% this leaves a gap because the flux departing is unknown
These changes, and the relatively low resolution of the initial measurements leave a notable gap in our understanding of the reactor system.
% the changes in the collimator are likely to have had a significant effect on both the flux(E,theta,etc.) and the absolute flux
The changes within this northeast beam port are likely to have a significant effect on both the spectral, angular, and radial characteristics of the flux, as well as its absolute magnitude.
% there is minimal simulation work and experimental work that has been documented that characterizes this
Minimal simulation and experimental work has been performed to characterize of this updated system.
% experiments have to do flux characterizations every time they experiment, and don't have a reliable source term or a good idea for the f/t ratio, which is important for testing fast detectors, etc.
Experimentalists currently are either required to do a form of flux characterization every time this beam port is used, or rely on a questionable dataset to complete calculations and simulations related to their devices.
% although you can use a meter to check the dose, there's still a radiological hazard anytime you deal with an unknown source
This is not to mention the potential radiological hazard associated with any partially unknown source of radiation.



\section{Justification and Outline of Conducted Research}
%% identify the importance of the proposed research
% it's important then, to do a full characterization of this beam port to fill this gap
It is important, then, to do a full characterization of the northeast beam port to fill this gap.
% a better understanding would increase our capabilities for experiments, inform experimental modeling, and improve the safety of beam port use
A better understand of this port would increase the experimental capability for the purpose of detector characterization, inform experimental modeling, and improve the safety of anyone using the northeast beam port.

%% state the research problem/questions
% the research question is, what is coming out of the beam port
Simply stated, this work seeks to understand the spectral properties of the neutrons departing from the northeast beam port.
% is it possible to build a model to simulate it, then conduct an experiment to validate that model
% is it possible to accurately capture the radial, spectral, and angular distributions of this flux
Is it possible to build a model to simulate the port's flux, including the radial, spectral, and angular distributions of this flux, and then conduct an experiment to validate that model?

%% state the research aims and/or research objectives
% the research aims to compute this flux, with all dimensions, with a very high fidelity
This research aims to compute this flux, with all mentioned dimensions with a very high fidelity and resolution.
% then, the goal is to verify the flux values using experimental spectrometric techniques
Then, the goal is to verify these flux values using experimental spectrometric techniques.

%% state the hypotheses
% its hypothesized that the beam will contain a significant contribution from fast neutrons
It has been hypothesized that this beam will contain a relatively significant contribution from fast neutrons because it, unlike other ports, penetrates the graphite reflector surrounding the core.
% the beam should be predominantly monodirectional
The neutrons departing from this beam should also be predominantly monodirectional.

%% outline the order of information in the thesis
% first, a review of neutron spectrum unfolding methods will be described
To achieve these goals and test the hypotheses, first, a review of useful neutron spectrometers and neutron spectrum deconvolution methods will be described in chapter 2.
% then, the whole modeling effort will be described, including results
Then, in chapter 3, the whole modeling effort conducted will be described, including any preliminary simulated results.
% following that, a description of the experimental campaign
% finally, a presentation of the unfolding results to show a set of solutions consistent with the data
Following that, a description of the experimental campaign used to validate these models will be presented in chapter 4 along with a series of unfolding results that show a set of solutions consistent with the experimental data.
Chapter 5 will include conclusions, contributions to the field, and proposed future work.


