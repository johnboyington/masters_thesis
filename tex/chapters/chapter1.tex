
\cleardoublepage


\chapter{Introduction and Background}



\section{Establish Territory}
%% state the general topic and give some background
% capabilities of nuclear technology is very powerful
% however, these great capabilities are often limited by our true understanding of the systems that underly them
% having methods to obtain accurate knowledge of how these systems operate and evolve raise this fundamental bar
% therefore, from essentially the beginning of nuclear science, measurement techniques have had an __ niche in supporting the development of the science and technology
% and today this is no different

%% provide a review of the literature related to the topic
% in the field of neutron detection, a huge body of scientific methods and technologies have developed that allow an experimentor to reliably and accurately obtain a most fundamental property of ___, the neutron spectrum
% active scintillation detectors, proton recoil detectors, and passive activation detectors represent a portion of the different way information on a neutron spectrum is obtained
% though these detectors allow us to find information on a neutronic system, a large field of mathematics have allowed even greater capabilities and even more information to be extracted
% spectral deconvolution/unfolding techniques have allowed a small series of measurements to be translated into a very large group neutron flux, a feat once considered impossible

%% define the terms and scope of the topic
% this field, neutron spectrometry, represents the field of research, within which the topics discussed in this thesis lie.
% by employing a variety of simulation, measurement, and unfolding techniques, the author will show how these components have fit together to ultimately widen the field of scientific knowledge



\section{Establish a Niche}
%% outline the current situation
% mention the mark ii reactor
% mention the beam ports
% initial flux characterization f/t ratio, flux
% since then, geometry of the beam port has changed significantly
% the core has also underwent change, fuel rod positions, burnup

%% evaluate the current situation (advantages/disadvantages) and identify the gap
% this leaves a gap because the flux departing is unknown
% the changes in the collimator are likely to have had a significant effect on both the flux(E,theta,etc.) and the absolute flux
% there is minimal simulation work and experimental work that has been documented that characterizes this
% experiments have to do flux characterizations every time they experiment, and don't have a reliable source term or a good idea for the f/t ratio, which is important for testing fast detectors, etc.
% although you can use a meter to check the dose, there's still a radiological hazard anytime you deal with an unknown source



\section{Introduce the Current Research}
%% identify the importance of the proposed research
% it's important then, to do a full characterization of this beam port to fill this gap
% a better understanding would increase our capabilities for experiments, inform experimental modeling, and improve the safety of beam port use

%% state the research problem/questions
% the research quesiton is, what is coming out of the beam port
% is it possible to build a model to simulate it, then conduct an experiment to validate that model
% is it possible to accurately capture the radial, spectral, and angular distributions of this flux

%% state the research aims and/or research objectives
% the research aims to compute this flux, with all dimensions, with a very high fidelity
% then, the goal is to verify the flux values using experimental spectrometric techniques

%% state the hypothese
% its hypothesized that the beam will contain a significant contribution from fast neutrons
% the beam should be predominantly monodirectional

%% outline the order of information in the thesis
% first, a review of neutron spectrum unfolding methods will be described
% then, the whole modeling effort will be described, including results
% following that, a description of the experimental campaign
% finally, a presentation of the unfolding results to show a set of solutions consistent with the data

%% outline the methodology
% this will be employ a myriad of techniques, monte carlo neutron transport, passive and active neutron detection and different math methods for the unfolding



