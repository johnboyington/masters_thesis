% +--------------------------------------------------------------------+
% | Abstract Page
% +--------------------------------------------------------------------+

\pagestyle{empty}
%\vspace{1cm}
\setlength{\baselineskip}{0.8cm}

%\indent

% +--------------------------------------------------------------------+
% | Enter the text of your abstract below.  There is no limit on the
% | number of words in your abstract.
% +--------------------------------------------------------------------+

% spectral characterization was done on nebp
A high-resolution, multi-dimensional flux characterization was performed for the Kansas State University TRIGA Mark II's Northeast Beam Port for the purpose of informing future experimental work, such as detector characterization, done using the beam.
% then i added nebp to the existing ksu triga model
First, the Beam Port geometry was added to the existing reactor model.
% fission rates were tallied within the core
Then the in-core fission rates were tallied using MCNP to provide a source term for the beam transport.
% advantg was applied
The program ADVANTG was used for automated generation of weight windows to accelerate the convergence of tallies in the beam port model.
% tally was collected at end of beam port
A tally was then collected at the end of the beam port and results are presented.

% this tally was used to produce response functions for different detectors
Results from this tally were used to produce simulated responses and response functions for two detectors: a gold foil-based passive spectrometer and a standard set of Bonner Spheres.
% experiment done to measure flux
An experiment was conducted with both measurement devices to obtain responses.
% post processing done, showing decent agreeance with simulated results
These measurements showed decent shape agreement and a small magnitude bias relative to the simulated results.
% final spectra were unfolded using methods described above and results presented
Finally, the measurements, simulated flux, and response functions were used to unfold a final set of spectra using three different unfolding techniques.
The Doroshenko directed divergence and Gravel (modified Sand-II) methods produced physically-realistic spectra which successfully fit the measured data, while MAXED did not.

